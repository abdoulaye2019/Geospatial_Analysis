% Options for packages loaded elsewhere
\PassOptionsToPackage{unicode}{hyperref}
\PassOptionsToPackage{hyphens}{url}
%
\documentclass[
]{article}
\usepackage{amsmath,amssymb}
\usepackage{lmodern}
\usepackage{iftex}
\ifPDFTeX
  \usepackage[T1]{fontenc}
  \usepackage[utf8]{inputenc}
  \usepackage{textcomp} % provide euro and other symbols
\else % if luatex or xetex
  \usepackage{unicode-math}
  \defaultfontfeatures{Scale=MatchLowercase}
  \defaultfontfeatures[\rmfamily]{Ligatures=TeX,Scale=1}
\fi
% Use upquote if available, for straight quotes in verbatim environments
\IfFileExists{upquote.sty}{\usepackage{upquote}}{}
\IfFileExists{microtype.sty}{% use microtype if available
  \usepackage[]{microtype}
  \UseMicrotypeSet[protrusion]{basicmath} % disable protrusion for tt fonts
}{}
\makeatletter
\@ifundefined{KOMAClassName}{% if non-KOMA class
  \IfFileExists{parskip.sty}{%
    \usepackage{parskip}
  }{% else
    \setlength{\parindent}{0pt}
    \setlength{\parskip}{6pt plus 2pt minus 1pt}}
}{% if KOMA class
  \KOMAoptions{parskip=half}}
\makeatother
\usepackage{xcolor}
\usepackage[margin=1in]{geometry}
\usepackage{color}
\usepackage{fancyvrb}
\newcommand{\VerbBar}{|}
\newcommand{\VERB}{\Verb[commandchars=\\\{\}]}
\DefineVerbatimEnvironment{Highlighting}{Verbatim}{commandchars=\\\{\}}
% Add ',fontsize=\small' for more characters per line
\usepackage{framed}
\definecolor{shadecolor}{RGB}{248,248,248}
\newenvironment{Shaded}{\begin{snugshade}}{\end{snugshade}}
\newcommand{\AlertTok}[1]{\textcolor[rgb]{0.94,0.16,0.16}{#1}}
\newcommand{\AnnotationTok}[1]{\textcolor[rgb]{0.56,0.35,0.01}{\textbf{\textit{#1}}}}
\newcommand{\AttributeTok}[1]{\textcolor[rgb]{0.77,0.63,0.00}{#1}}
\newcommand{\BaseNTok}[1]{\textcolor[rgb]{0.00,0.00,0.81}{#1}}
\newcommand{\BuiltInTok}[1]{#1}
\newcommand{\CharTok}[1]{\textcolor[rgb]{0.31,0.60,0.02}{#1}}
\newcommand{\CommentTok}[1]{\textcolor[rgb]{0.56,0.35,0.01}{\textit{#1}}}
\newcommand{\CommentVarTok}[1]{\textcolor[rgb]{0.56,0.35,0.01}{\textbf{\textit{#1}}}}
\newcommand{\ConstantTok}[1]{\textcolor[rgb]{0.00,0.00,0.00}{#1}}
\newcommand{\ControlFlowTok}[1]{\textcolor[rgb]{0.13,0.29,0.53}{\textbf{#1}}}
\newcommand{\DataTypeTok}[1]{\textcolor[rgb]{0.13,0.29,0.53}{#1}}
\newcommand{\DecValTok}[1]{\textcolor[rgb]{0.00,0.00,0.81}{#1}}
\newcommand{\DocumentationTok}[1]{\textcolor[rgb]{0.56,0.35,0.01}{\textbf{\textit{#1}}}}
\newcommand{\ErrorTok}[1]{\textcolor[rgb]{0.64,0.00,0.00}{\textbf{#1}}}
\newcommand{\ExtensionTok}[1]{#1}
\newcommand{\FloatTok}[1]{\textcolor[rgb]{0.00,0.00,0.81}{#1}}
\newcommand{\FunctionTok}[1]{\textcolor[rgb]{0.00,0.00,0.00}{#1}}
\newcommand{\ImportTok}[1]{#1}
\newcommand{\InformationTok}[1]{\textcolor[rgb]{0.56,0.35,0.01}{\textbf{\textit{#1}}}}
\newcommand{\KeywordTok}[1]{\textcolor[rgb]{0.13,0.29,0.53}{\textbf{#1}}}
\newcommand{\NormalTok}[1]{#1}
\newcommand{\OperatorTok}[1]{\textcolor[rgb]{0.81,0.36,0.00}{\textbf{#1}}}
\newcommand{\OtherTok}[1]{\textcolor[rgb]{0.56,0.35,0.01}{#1}}
\newcommand{\PreprocessorTok}[1]{\textcolor[rgb]{0.56,0.35,0.01}{\textit{#1}}}
\newcommand{\RegionMarkerTok}[1]{#1}
\newcommand{\SpecialCharTok}[1]{\textcolor[rgb]{0.00,0.00,0.00}{#1}}
\newcommand{\SpecialStringTok}[1]{\textcolor[rgb]{0.31,0.60,0.02}{#1}}
\newcommand{\StringTok}[1]{\textcolor[rgb]{0.31,0.60,0.02}{#1}}
\newcommand{\VariableTok}[1]{\textcolor[rgb]{0.00,0.00,0.00}{#1}}
\newcommand{\VerbatimStringTok}[1]{\textcolor[rgb]{0.31,0.60,0.02}{#1}}
\newcommand{\WarningTok}[1]{\textcolor[rgb]{0.56,0.35,0.01}{\textbf{\textit{#1}}}}
\usepackage{graphicx}
\makeatletter
\def\maxwidth{\ifdim\Gin@nat@width>\linewidth\linewidth\else\Gin@nat@width\fi}
\def\maxheight{\ifdim\Gin@nat@height>\textheight\textheight\else\Gin@nat@height\fi}
\makeatother
% Scale images if necessary, so that they will not overflow the page
% margins by default, and it is still possible to overwrite the defaults
% using explicit options in \includegraphics[width, height, ...]{}
\setkeys{Gin}{width=\maxwidth,height=\maxheight,keepaspectratio}
% Set default figure placement to htbp
\makeatletter
\def\fps@figure{htbp}
\makeatother
\setlength{\emergencystretch}{3em} % prevent overfull lines
\providecommand{\tightlist}{%
  \setlength{\itemsep}{0pt}\setlength{\parskip}{0pt}}
\setcounter{secnumdepth}{-\maxdimen} % remove section numbering
\ifLuaTeX
  \usepackage{selnolig}  % disable illegal ligatures
\fi
\IfFileExists{bookmark.sty}{\usepackage{bookmark}}{\usepackage{hyperref}}
\IfFileExists{xurl.sty}{\usepackage{xurl}}{} % add URL line breaks if available
\urlstyle{same} % disable monospaced font for URLs
\hypersetup{
  pdftitle={Analyse Spatiale Dans R},
  pdfauthor={Abdoulaye Leye (GiS, Data Analyst Specialist)},
  hidelinks,
  pdfcreator={LaTeX via pandoc}}

\title{Analyse Spatiale Dans R}
\author{Abdoulaye Leye (GiS, Data Analyst Specialist)}
\date{2023-02-12}

\begin{document}
\maketitle

\hypertarget{utilsation-des-packages-ggplot2-et-sf}{%
\subsection{Utilsation des Packages ggplot2 et
sf}\label{utilsation-des-packages-ggplot2-et-sf}}

L'analyse de données spatiales consiste à explorer, visualiser et
modéliser des données géographiques pour en extraire des informations
utiles. R et RStudio sont des outils puissants pour l'analyse de données
spatiales en raison de la disponibilité de packages tels que
\texttt{sf}, \texttt{sp}, \texttt{raster} et \texttt{leaflet} qui
permettent de travailler facilement avec des données géospatiales.

Pour commencer à travailler avec des données spatiales dans R, vous
devez d'abord charger les packages nécessaires et les données en
utilisant les commandes suivantes :

\begin{itemize}
\tightlist
\item
  \texttt{library(ggplot2)} et \texttt{library(sf)} ces deux packages
  seront utilisés dans cet article.Rmarkdown
  \includegraphics[width=0.05\textwidth,height=\textheight]{rmarkdown_logo.png}
  qui est language accesseble et trés simple pour sera utilisé pour
  réaliser les travaux
\end{itemize}

\begin{center}\rule{0.5\linewidth}{0.5pt}\end{center}

\hypertarget{les-bibliothuxe8ques-ggplot2-et-sf}{%
\subsection{\texorpdfstring{Les bibliothèques \textbf{ggplot2} et
\textbf{sf}:}{Les bibliothèques ggplot2 et sf:}}\label{les-bibliothuxe8ques-ggplot2-et-sf}}

\hypertarget{le-package-gglot2}{%
\subsubsection{\texorpdfstring{Le package
\textbf{gglot2}:}{Le package gglot2:}}\label{le-package-gglot2}}

\href{\textquotesingle{}https://ggplot2.tidyverse.org/\textquotesingle{}}{ggplot2}
est un package populaire pour la visualisation de données dans R. Il
fournit une interface de haut niveau pour créer des graphiques complexes
à partir de données tabulaires. La force de \texttt{ggplot2} réside dans
sa capacité à produire des graphiques élégants et informatifs en
utilisant un langage simple et intuitif pour décrire les différents
éléments d'un graphique. \url{https://ggplot2.tidyverse.org/}.

Le package utilise un système de ``grammaire de la géométrie'' pour
décrire les différentes couches d'un graphique. Chaque couche représente
un aspect différent des données, tels que les points, les barres, les
lignes, les aires de remplissage, etc. Vous pouvez facilement ajouter,
supprimer et personnaliser ces couches pour obtenir un graphique qui
répond à vos besoins.

Pour utiliser \texttt{ggplot2}, vous devez d'abord charger le package :

\begin{Shaded}
\begin{Highlighting}[]
\FunctionTok{library}\NormalTok{(ggplot2)}
\end{Highlighting}
\end{Shaded}

\hypertarget{le-package-sf}{%
\subsubsection{\texorpdfstring{Le Package
\textbf{Sf}:}{Le Package Sf:}}\label{le-package-sf}}

\href{\textquotesingle{}https://cran.r-project.org/web/packages/sf/index.html\textquotesingle{}}{sf}
est un package de R pour la manipulation de données géospatiales
vectorielles. Il fournit un ensemble de fonctions pour travailler avec
des données géospatiales en utilisant le format de données simplifié
(Simple Features), qui est un standard industriel pour les données
géospatiales vectorielles.

Avec sf, vous pouvez facilement créer, manipuler et visualiser des
données géospatiales vectorielles. Il vous permet également de
travailler avec des données géospatiales dans un format R standard, ce
qui vous permet de bénéficier de l'intégration avec d'autres packages R
pour la visualisation, la modélisation et l'analyse de données.

Pour utiliser \texttt{sf}, vous devez d'abord charger le package :

\begin{Shaded}
\begin{Highlighting}[]
\FunctionTok{library}\NormalTok{(sf)}
\end{Highlighting}
\end{Shaded}

Ensuite, vous pouvez charger des données géospatiales vectorielles à
partir de différents formats de fichiers, tels que les fichiers
\texttt{Shapefile}, les fichiers \texttt{GeoJSON} et les fichiers
\texttt{CSV}. Par exemple, pour charger des données depuis un fichier
\texttt{Shapefile}, vous pouvez utiliser le code suivant :

\begin{Shaded}
\begin{Highlighting}[]
\NormalTok{communes }\OtherTok{\textless{}{-}} \FunctionTok{read\_sf}\NormalTok{(}\StringTok{"../shapefiles/communes\_senegal\_2015.shp"}\NormalTok{)}
\NormalTok{regions }\OtherTok{\textless{}{-}} \FunctionTok{read\_sf}\NormalTok{(}\StringTok{"../shapefiles/gadm36\_SEN\_1.shp"}\NormalTok{)}
\end{Highlighting}
\end{Shaded}

Une fois que vous avez chargé les données, vous pouvez les visualiser en
utilisant la fonction ggplot() de la bibliothèque ggplot2 et en ajoutant
une couche géospatiale en utilisant la fonction geom\_sf()

Ici on peut visualiser les données pour avoir une aperçue gloables des
différentes variables de la tables attributaire. Example les communes du
Sénégal.

\begin{Shaded}
\begin{Highlighting}[]
\FunctionTok{plot}\NormalTok{(regions)}
\end{Highlighting}
\end{Shaded}

\includegraphics{index_files/figure-latex/unnamed-chunk-2-1.pdf}

Une fois les données chargées, vous pouvez les manipuler \texttt{sf} et
les visualiser avec \texttt{ggplot2}. Par exemple, pour visualiser les
données sur une carte, vous pouvez utiliser le code suivant :

\texttt{sf} offre également un large éventail de fonctions pour
manipuler les données géospatiales, telles que la transformation de
coordonnées, la projection, la dissociation et l'agrégation des données,
la création de buffers, l'intersection et la fusion de données, etc.

\emph{Visualisation simple de la couche régions du Sénégal.}

\begin{Shaded}
\begin{Highlighting}[]
\FunctionTok{ggplot}\NormalTok{() }\SpecialCharTok{+}
  \FunctionTok{geom\_sf}\NormalTok{(}\AttributeTok{data =}\NormalTok{ regions)}
\end{Highlighting}
\end{Shaded}

\includegraphics{index_files/figure-latex/unnamed-chunk-3-1.pdf}

\emph{Définition des intervalles et labelisation de la légende.}

\begin{Shaded}
\begin{Highlighting}[]
\NormalTok{communes}\SpecialCharTok{$}\NormalTok{Density }\OtherTok{\textless{}{-}} \FunctionTok{cut}\NormalTok{(communes}\SpecialCharTok{$}\NormalTok{TOTAL, }\AttributeTok{breaks =} \FunctionTok{c}\NormalTok{(}\DecValTok{0}\NormalTok{,}\DecValTok{2000}\NormalTok{,}\DecValTok{10000}\NormalTok{,}\DecValTok{20000}\NormalTok{,}\DecValTok{30000}\NormalTok{,}\ConstantTok{Inf}\NormalTok{),}
                       \AttributeTok{labels =} \FunctionTok{c}\NormalTok{(}\StringTok{\textquotesingle{}\textless{} 1000\textquotesingle{}}\NormalTok{, }\StringTok{\textquotesingle{}1000{-}10000\textquotesingle{}}\NormalTok{, }\StringTok{\textquotesingle{}10000{-}20000\textquotesingle{}}\NormalTok{, }\StringTok{\textquotesingle{}20000{-}30000\textquotesingle{}}\NormalTok{, }\StringTok{\textquotesingle{}\textgreater{} 30000\textquotesingle{}}\NormalTok{))}
\end{Highlighting}
\end{Shaded}

\hypertarget{ruxe9aliser-une-carte-avec-la-fonction-geom_sf}{%
\subsubsection{\texorpdfstring{Réaliser une Carte avec la fonction
\textbf{geom\_sf}}{Réaliser une Carte avec la fonction geom\_sf}}\label{ruxe9aliser-une-carte-avec-la-fonction-geom_sf}}

\texttt{geom\_sf()} est une géométrie dans le package \texttt{ggplot2}
qui permet de visualiser des données spatiales dans \textbf{R}. Il est
spécialement conçu pour travailler avec des données géospatiales
structurées dans le format sf (simple features) et est conçu pour
remplacer la géométrie \texttt{geom\_map()} utilisée dans les versions
antérieures de \texttt{ggplot2}.

Pour utiliser \texttt{geom\_sf()}, vous devez d'abord avoir des données
géospatiales sous forme de sf object, puis les charger dans ggplot2 en
utilisant la fonction \texttt{ggplot()}.Example d'utilisation:

\begin{Shaded}
\begin{Highlighting}[]
\FunctionTok{ggplot}\NormalTok{()}\SpecialCharTok{+}
  \FunctionTok{geom\_sf}\NormalTok{(}\FunctionTok{aes}\NormalTok{(}\AttributeTok{fill =}\NormalTok{ Density),}\AttributeTok{color =} \StringTok{\textquotesingle{}transparent\textquotesingle{}}\NormalTok{, }\AttributeTok{data =}\NormalTok{ communes)}\SpecialCharTok{+}
  \FunctionTok{geom\_sf}\NormalTok{(}\AttributeTok{fill =} \StringTok{\textquotesingle{}transparent\textquotesingle{}}\NormalTok{, }\AttributeTok{color =} \StringTok{\textquotesingle{}white\textquotesingle{}}\NormalTok{, }\AttributeTok{data =}\NormalTok{ regions) }\SpecialCharTok{+}
  \FunctionTok{scale\_fill\_viridis\_d}\NormalTok{(}\AttributeTok{name =} \StringTok{"Population/Communes"}\NormalTok{,}
                       \AttributeTok{guide =} \FunctionTok{guide\_legend}\NormalTok{(}
                         \AttributeTok{direction =} \StringTok{\textquotesingle{}horizontal\textquotesingle{}}\NormalTok{,}
                         \AttributeTok{title.position =} \StringTok{\textquotesingle{}top\textquotesingle{}}\NormalTok{,}
                         \AttributeTok{title.hjust =}\NormalTok{ .}\DecValTok{5}\NormalTok{,}
                         \AttributeTok{label.hjust =}\NormalTok{ .}\DecValTok{5}\NormalTok{,}
                         \AttributeTok{label.position =} \StringTok{\textquotesingle{}bottom\textquotesingle{}}\NormalTok{,}
                         \AttributeTok{keywidth =} \DecValTok{3}\NormalTok{,}
                         \AttributeTok{keyheight =}\NormalTok{ .}\DecValTok{5}
\NormalTok{                       ))}\SpecialCharTok{+}
  \FunctionTok{labs}\NormalTok{(}\AttributeTok{title =} \StringTok{"Démographie du Sénégal"}\NormalTok{,}
       \AttributeTok{subtitle =} \StringTok{"Répartition de population du Sénégal par Communes"}\NormalTok{,}
       \AttributeTok{caption =} \FunctionTok{c}\NormalTok{(}\StringTok{"Sources: ASER, ANSD National Agence Of Statistique and Demography 2015"}\NormalTok{))}\SpecialCharTok{+}
  \FunctionTok{theme\_void}\NormalTok{()}\SpecialCharTok{+}
  \FunctionTok{theme}\NormalTok{(}\AttributeTok{title =} \FunctionTok{element\_text}\NormalTok{(}\AttributeTok{face =} \StringTok{\textquotesingle{}bold\textquotesingle{}}\NormalTok{),}
        \AttributeTok{legend.position =} \StringTok{\textquotesingle{}bottom\textquotesingle{}}\NormalTok{)}
\end{Highlighting}
\end{Shaded}

\includegraphics{index_files/figure-latex/unnamed-chunk-5-1.pdf}

\hypertarget{analyse-exploratoire-eda}{%
\subsubsection{\texorpdfstring{Analyse Exploratoire
\textbf{EDA}}{Analyse Exploratoire EDA}}\label{analyse-exploratoire-eda}}

L'analyse exploratoire des données (EDA en anglais) est une étape
importante dans le processus d'analyse des données, qui consiste à
examiner les données pour découvrir les relations, les modèles et les
tendances cachées. L'objectif principal de l'EDA est de comprendre les
données sous différents angles, en utilisant des techniques graphiques
et statistiques pour mieux les visualiser et les comprendre. Ici il est
possible d'utiliser la fonction \texttt{st\_drop\_geometry} pour avoir
un DataFrame avec la table attributaire de notre jeux de données
\textbf{Shapefile}

\begin{Shaded}
\begin{Highlighting}[]
\FunctionTok{library}\NormalTok{(DT)}
\NormalTok{data }\OtherTok{\textless{}{-}} \FunctionTok{st\_drop\_geometry}\NormalTok{(communes)}
\NormalTok{data }\SpecialCharTok{\%\textgreater{}\%} \FunctionTok{select}\NormalTok{(REG, DEPT, CCRCA, SUP\_HA, Milieu, Masculin, Feminin, TOTAL) }\SpecialCharTok{\%\textgreater{}\%}
  \FunctionTok{head}\NormalTok{(}\DecValTok{5}\NormalTok{) }\SpecialCharTok{\%\textgreater{}\%} 
  \FunctionTok{datatable}\NormalTok{()}
\end{Highlighting}
\end{Shaded}

L'agrégation de données est le processus de regroupement de données
selon certaines caractéristiques ou dimensions et la création de
sommaires ou d'informations agrégées à partir de ces données. Ce
processus est souvent utilisé pour résumer de grandes quantités de
données en informations plus facilement compréhensibles et
exploitables.Dans cet example il est possible d'utiliser les fonctions
\texttt{group\_by} et \texttt{summarise} dans la package \texttt{dplyr}
pour faire une aggrégation des donées faire l'intégration avec la
bibliothéque \texttt{DT} afin d'avoir une idée plus précise de notre
\textbf{DataFrame}.

\begin{Shaded}
\begin{Highlighting}[]
\FunctionTok{library}\NormalTok{(DT)}
\NormalTok{data }\SpecialCharTok{\%\textgreater{}\%} 
  \FunctionTok{group\_by}\NormalTok{(REG) }\SpecialCharTok{\%\textgreater{}\%} 
  \FunctionTok{summarise}\NormalTok{(}\AttributeTok{Population =} \FunctionTok{sum}\NormalTok{(TOTAL)) }\SpecialCharTok{\%\textgreater{}\%} 
  \FunctionTok{datatable}\NormalTok{()}
\end{Highlighting}
\end{Shaded}

\hypertarget{graphique-en-colonne-verticale-avec-ggplot2}{%
\subsubsection{\texorpdfstring{Graphique en Colonne Verticale avec
\texttt{ggplot2}}{Graphique en Colonne Verticale avec ggplot2}}\label{graphique-en-colonne-verticale-avec-ggplot2}}

\begin{Shaded}
\begin{Highlighting}[]
\FunctionTok{options}\NormalTok{(}\AttributeTok{scipen =} \DecValTok{999}\NormalTok{)}
\FunctionTok{library}\NormalTok{(scales)}
\NormalTok{data }\SpecialCharTok{\%\textgreater{}\%} 
  \FunctionTok{group\_by}\NormalTok{(REG) }\SpecialCharTok{\%\textgreater{}\%} 
  \FunctionTok{summarise}\NormalTok{(}\AttributeTok{Population =} \FunctionTok{sum}\NormalTok{(TOTAL)) }\SpecialCharTok{\%\textgreater{}\%} 
  \FunctionTok{ggplot}\NormalTok{(}\FunctionTok{aes}\NormalTok{(}\AttributeTok{x =} \FunctionTok{reorder}\NormalTok{(REG,}\SpecialCharTok{{-}}\NormalTok{Population),}\AttributeTok{y =}\NormalTok{  Population)) }\SpecialCharTok{+}
  \FunctionTok{geom\_col}\NormalTok{(}\AttributeTok{color =} \StringTok{\textquotesingle{}tomato\textquotesingle{}}\NormalTok{, }\AttributeTok{fill =} \StringTok{\textquotesingle{}tomato\textquotesingle{}}\NormalTok{) }\SpecialCharTok{+}
  \FunctionTok{theme\_minimal}\NormalTok{() }\SpecialCharTok{+}
  \FunctionTok{labs}\NormalTok{(}\AttributeTok{title =} \StringTok{"Population par Régions"}\NormalTok{,}
       \AttributeTok{subtitle =} \StringTok{"Répartition de la population du Sénégal par région en 2015"}\NormalTok{,}
       \AttributeTok{x =} \StringTok{"Population"}\NormalTok{,}
       \AttributeTok{y =} \StringTok{"Régions"}\NormalTok{) }\SpecialCharTok{+}
  \FunctionTok{theme}\NormalTok{(}\AttributeTok{axis.text.x =} \FunctionTok{element\_text}\NormalTok{(}\AttributeTok{angle =} \DecValTok{90}\NormalTok{, }\AttributeTok{vjust =} \FloatTok{0.5}\NormalTok{, }\AttributeTok{hjust=}\DecValTok{1}\NormalTok{))}
\end{Highlighting}
\end{Shaded}

\includegraphics{index_files/figure-latex/unnamed-chunk-8-1.pdf}

\hypertarget{la-fonction-pivot_longer}{%
\subsubsection{\texorpdfstring{La fonction
\texttt{pivot\_longer}}{La fonction pivot\_longer}}\label{la-fonction-pivot_longer}}

pivot\_longer() est une fonction de transformation de données dans le
package tidyr de R qui permet de transformer un jeu de données d'un
format large en un format long en fusionnant plusieurs colonnes en une
seule colonne avec des noms de colonne séparés par un caractère commun.

\begin{Shaded}
\begin{Highlighting}[]
\NormalTok{data\_long }\OtherTok{\textless{}{-}}\NormalTok{ data }\SpecialCharTok{\%\textgreater{}\%} 
  \FunctionTok{group\_by}\NormalTok{(REG) }\SpecialCharTok{\%\textgreater{}\%} 
  \FunctionTok{summarise}\NormalTok{(}\AttributeTok{Masculin =} \FunctionTok{sum}\NormalTok{(Masculin), }\AttributeTok{Feminin =} \FunctionTok{sum}\NormalTok{(Feminin)) }\SpecialCharTok{\%\textgreater{}\%}
  \FunctionTok{pivot\_longer}\NormalTok{(}\SpecialCharTok{!}\NormalTok{REG, }\AttributeTok{names\_to =} \StringTok{"Groupe"}\NormalTok{, }\AttributeTok{values\_to =} \StringTok{"Valeur"}\NormalTok{) }\SpecialCharTok{\%\textgreater{}\%}
  \FunctionTok{ungroup}\NormalTok{() }\SpecialCharTok{\%\textgreater{}\%}
  \FunctionTok{arrange}\NormalTok{(Valeur) }\SpecialCharTok{\%\textgreater{}\%} 
  \FunctionTok{mutate}\NormalTok{(}\AttributeTok{order =} \FunctionTok{row\_number}\NormalTok{())}
\end{Highlighting}
\end{Shaded}

\hypertarget{graphique-en-colonne-verticale-groupuxe9-avec-ggplot2}{%
\subsubsection{\texorpdfstring{Graphique en Colonne Verticale Groupé
avec
\texttt{ggplot2}}{Graphique en Colonne Verticale Groupé avec ggplot2}}\label{graphique-en-colonne-verticale-groupuxe9-avec-ggplot2}}

\begin{Shaded}
\begin{Highlighting}[]
\FunctionTok{ggplot}\NormalTok{(data\_long, }\FunctionTok{aes}\NormalTok{(}\AttributeTok{x =} \FunctionTok{reorder}\NormalTok{(REG, }\SpecialCharTok{{-}}\NormalTok{Valeur), }\AttributeTok{fill=}\NormalTok{Groupe, }\AttributeTok{y=}\NormalTok{Valeur)) }\SpecialCharTok{+} 
  \FunctionTok{geom\_bar}\NormalTok{(}\AttributeTok{position=}\StringTok{"dodge"}\NormalTok{, }\AttributeTok{stat=}\StringTok{"identity"}\NormalTok{, ) }\SpecialCharTok{+}
  \FunctionTok{scale\_fill\_viridis\_d}\NormalTok{() }\SpecialCharTok{+}
  \FunctionTok{theme\_minimal}\NormalTok{() }\SpecialCharTok{+}
  \FunctionTok{labs}\NormalTok{(}\AttributeTok{title =} \StringTok{"Population par Régions et par Sexe"}\NormalTok{,}
       \AttributeTok{subtitle =} \StringTok{"Répartition de la population du Sénégal par région et par sexe en 2015"}\NormalTok{,}
       \AttributeTok{x =} \StringTok{"Régions"}\NormalTok{,}
       \AttributeTok{y =} \StringTok{"Population"}\NormalTok{) }\SpecialCharTok{+}
  \FunctionTok{theme}\NormalTok{(}\AttributeTok{axis.text.x =} \FunctionTok{element\_text}\NormalTok{(}\AttributeTok{angle =} \DecValTok{90}\NormalTok{, }\AttributeTok{vjust =} \FloatTok{0.5}\NormalTok{, }\AttributeTok{hjust=}\DecValTok{1}\NormalTok{))}
\end{Highlighting}
\end{Shaded}

\includegraphics{index_files/figure-latex/unnamed-chunk-10-1.pdf}

\hypertarget{conclusion}{%
\subsubsection{Conclusion}\label{conclusion}}

R et RStudio sont des outils puissants pour l'analyse de données
spatiales, et il existe de nombreux packages disponibles pour vous aider
à exploiter pleinement les capacités de ces outils. Il est donc
important de prendre le temps d'apprendre à utiliser ces packages et de
comprendre les concepts sous-jacents pour pouvoir réaliser des analyses
spatiales efficaces.

\end{document}
